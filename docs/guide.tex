Title: Web-based todo program
Author: Jeff Schaller
Date: September 22-24, 2000
	revised November 4, 2000
	revised November 3, 2001 - minor wording changes

--------
Overview
--------
	This program is a set of CGI scripts written in perl. The first
program, wt_display, has: a form for adding new entries, a table of
options to filter the display, and finally the list of the entries. The
second program, wt_edit, allows updates to an entry. A third file,
utilities.pl, contains common subroutines.

--------------------
History & background
--------------------
	I had written a command-line program called 'jtodo' in order to
manage my tasks. I had thought about making a web-based version, but lacked
sufficient motivation. When another person joined my team at work, it
became necessary to consolidate our task lists.  Being a web-oriented team, 
I resurrected my original plans and created the program.
	Revisionist history: it turns out that having a title of "Jeff's
Awesome Web-based Program" has had some sort of inhibitory effect on the
team's usage of the program, so it's turned out that I'm the only one
actually using it -- and I've been using it daily for almost an entire 
year, now.

My original concerns revolved around security.  Web-based applications have
two huge problems here: authentication of users and protection of data. I
put an ACL on the directory which contains the CGI scripts to "solve" the 
authentication problem.  As far as protection of the data: the program has
a small number of known users, several of the fields have maximum-length
restraints, and the parts of the program which read and write the datafile
use advisory locking (the best that can be done). Both scripts use perl's 
"-w" flag (which warns about possible errors) as well as "use strict", 
which imports more restrictions on variable and subroutine usage.
Additionally, the most damage that can be done from within the program is
to either add bogus description entries or to change the percent-complete
or priority fields around. The program logs all of the changes, so it would
be easy to undo them.
BUT, for personal usage, I would strongly suggest that the programs not 
be accessible by random people on the Internet.

-----------
Usage guide
-----------
wt_display
	Adding an item
		The form supplies a mostly-blank form for adding new entries.
		It supplies a default "percent-complete" value of 0 and a "priority"
		of 50.  ** no longer true -- You must type at least one character in the "description"
		field in order for the entry to be added **.  Invalid characters in
		either the priority or percent-complete field will also cause the
		addition to fail. The priority and percent-complete fields must
		be numbers between 0 and 100, inclusive.  The "who" field is
		currently limited to 20 characters. The "when" field is currently
		limited to 12 characters.  Click "Add entry" to add it when you're
		ready. The fields in this area are /not/ "sticky" like most fields
		in CGI scripts -- this is a feature. It makes adding new entries
		easier by clearing out the old entry's information.

	Tweaking the filters
		There are three knobs you can twiddle: sorting, filtering on the
		"who" field, and deciding whether to display 100% complete items.

		There are three ways to sort the entries: by weighted priority,
		percent-complete, or priority.  A weighted-priority sort works by
		multiplying an item's priority by 100 minus percent-complete.
		This has the effect of putting higher-priority, less-complete items
		at the top. This is the default sort.  Sorting by percent-complete
		and priority are straightforward.  A checkbox lets you reverse the
		sense of the sort.

		If you're just interested in items which have a certain person
		working on them, enter that name in the "who" filter. Leaving this
		field blank will skip the name-filtering. It is blank by default.

		A checkbox exists to let you choose whether or not to display
		items that have been completed (have 100 in the percent-complete
		field). Leaving this box unchecked reduces screen clutter. It is
		unchecked by default.

		After selecting the options you want (if they're different from
		the defaults), click "Redisplay entries" to update the list.
		The above options are "sticky"; that is, after selecting them and
		clicking "Redisplay", the same options will be set. This is an
		intended feature :)

	The entries
		Here, the entries are listed in a table according to the filters
		set above.  Unexciting, except for the "Id" field, which contains
		a hyperlink to the wt_edit script for that entry. Segue, anyone?

wt_edit
	After clicking on a certain entry's Id, wt_edit displays a form to let you
	edit certain fields. You cannot edit an entry's description -- it is
	displayed above a text field where you can enter additional information.

	The "who" and "when" fields, if they were previously blank, will now show
	"&nbsp;" instead.  This is not a bug, but rather a work-around for
	netscape's odd behavior when displaying blank information in a table.
	If the field was /really/ blank, netscape stops drawing the table border
	around the field; with the "&nbsp;" in there, netscape will draw the
	border and simply fill the field with a space. Using an actual space (" ")
	doesn't work, unfortunately.

	Clicking on "Reset Values" will take things back the way they were.
	Clicking "Submit" will modify the entry.  Note that if any illegal values
	for percent-complete or priority exist, the field will default back to its
	previous value. After submitting the form, you are taken back to wt_display.
	A hyperlink exists which will take you back, as well.

------------
Installation
------------
See the README file
